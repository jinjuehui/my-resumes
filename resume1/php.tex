\documentclass{resume}
\usepackage{zh_CN-Adobefonts_external}
\usepackage{linespacing_fix} % disable extra space before next section
\usepackage{cite}

\begin{document}
\pagestyle{empty}

\name{高正炎}
\job{后端开发实习生}
\contactInfo{{moyishizhe@gmail.com}}{18404968744}{http://dbqf.xyz}
\section{项目经历}
\datedsubsection{\textbf{猎鹰Docker云系统安全培训平台} , \href{http://www.lysec.org}{网址}}{2016 年7 月 -- 2016 年8 月}
\begin{onehalfspacing}
负责整体架构及后端实现
\begin{itemize}
  \item 利用 Docker 虚拟化技术复现漏洞,实现了多用户的隔离环境。
  \item 使用 etcd、confd、nginx、Docker、Phalcon等技术栈搭建平台环境。
  \item 构建了多个存在漏洞的镜像。
\end{itemize}
\end{onehalfspacing}

\datedsubsection{\textbf{解魔方机器人} , \href{https://github.com/DigDream/RubiksCubeRobot}{GitHub}, \href{http://v.youku.com/v_show/id_XMTQ1NjExMTk3Mg==.html}{优酷}}{2015 年3 月 -- 2015 年5 月}
\role{设计、开发}{Android, arduino}
\begin{onehalfspacing}
机器人硬件部分负责复原、打乱,识别部分采用手机APP摄像头扫描。
\begin{itemize}
  \item 与网上大部分的相比,实现了一种更加稳定的结构。在平台上,后来用到了ARM,遇到了多路输出pwm的问题,经过耐心的查找资料,调试,后来在比赛中采用了定时器输出pwm的方式。
  \item 利用OpenCV高效地进行颜色识别,遇到了环境强光减弱识别效果的问题,所以将识别图像由RGB模式转换到HSV模式。
  \item Android客户端集成了计时器,打乱公式,复原纪录,魔坛资讯等实用功能
\end{itemize}
\end{onehalfspacing}

\section{IT 技能}
\begin{onehalfspacing}
\begin{itemize}[parsep=0.5ex]
  \item 熟悉 PHP 及 PHP 基本框架 Phalcon、ThinkPHP,了解 Python 开发
  \item 熟悉 Android 安全及开发
  \item 数据库: MySQL > Mongodb > Redis
  \item 熟悉基本渗透测试
  \item 熟悉 Docker 及 DevOps 基本实践
\end{itemize}
\end{onehalfspacing}

\section{获奖情况}
\datedline{解魔方机器人, \textit{国家二等奖}, 全国第十一届博创杯比赛}{2015 年7 月}

\section{实践经历}
\datedline{创办并运营CSDN高校俱乐部社团, 并获优秀社团}{2013 年 -- 2015 年}

\section{教育背景}
\datedsubsection{\textbf{山西农业大学}}{2013 -- 至今}
\textit{在读本科}\ 软件工程, 预计 2017 年 6 月毕业

\section{其他}
\begin{itemize}[parsep=0.5ex]
  \item 技术博客: \href{http://dbqf.xyz}{http://dbqf.xyz}
  \item GitHub: \href{https://github.com/dubuqingfeng}{https://github.com/dubuqingfeng}
  \item 微信: dubuqingfeng
\end{itemize}

\end{document}
